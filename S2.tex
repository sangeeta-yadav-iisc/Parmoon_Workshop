 
%%%%%%%%%%%%%%%%%%%%%%%%%%%%%%%%%%%%%%%%%%%%%%%%%%%%%%%%%%%%
%%  This Beamer template was created by Cameron Bracken.
%%  Anyone can freely use or modify it for any purpose
%%  without attribution.
%%
%%  Last Modified: January 9, 2009
%%

\documentclass[xcolor=x11names,compress]{beamer}
% \documentclass[xcolor=x11names,compress,handout]{beamer}

%% General document %%%%%%%%%%%%%%%%%%%%%%%%%%%%%%%%%%
\usepackage{graphicx}
\usepackage{tikz}
\usetikzlibrary{decorations.fractals}
%%%%%%%%%%%%%%%%%%%%%%%%%%%%%%%%%%%%%%%%%%%%%%%%%%%%%%


%% Beamer Layout %%%%%%%%%%%%%%%%%%%%%%%%%%%%%%%%%%
\useoutertheme[subsection=false,shadow]{miniframes}
\useinnertheme{default}
\usefonttheme{serif}
\usepackage{palatino}

\setbeamerfont{title like}{shape=\scshape}
\setbeamerfont{frametitle}{shape=\scshape}

\setbeamercolor*{lower separation line head}{bg=DeepSkyBlue4} 
\setbeamercolor*{normal text}{fg=black,bg=white} 
\setbeamercolor*{alerted text}{fg=red} 
\setbeamercolor*{example text}{fg=black} 
\setbeamercolor*{structure}{fg=black} 
 
\setbeamercolor*{palette tertiary}{fg=black,bg=black!10} 
\setbeamercolor*{palette quaternary}{fg=black,bg=black!10} 

\renewcommand{\(}{\begin{columns}}
\renewcommand{\)}{\end{columns}}
\newcommand{\<}[1]{\begin{column}{#1}}
\renewcommand{\>}{\end{column}}
%%%%%%%%%%%%%%%%%%%%%%%%%%%%%%%%%%%%%%%%%%%%%%%%%%



\usepackage{pgf,pgfarrows,pgfnodes,pgfautomata,pgfheaps,pgfshade}
\usepackage{amsmath,amsfonts,amssymb,graphicx,epsfig}
\usepackage[latin1]{inputenc}
\usepackage{colortbl}
\usepackage[english]{babel}
\usepackage{graphics}
\usepackage{graphicx,amsbsy,amsmath,amsfonts}
%\usepackage{algorithm}
\usepackage{times}
\usepackage{listings}
\usepackage{fancyvrb}
% \usepackage{tabu}
% \usepackage{animate}
%\usepackage{capt-of}
% \newtheorem{assumption}{Assumption}

\def\ds{\displaystyle}
\newcommand{\dive}{\nabla\cdot}
\newcommand{\bD}{{\bf D}}
\newcommand{\bu}{{\bf u}}
\newcommand{\bv}{{\bf v}}
\newcommand{\bw}{{\bf w}}
\newcommand{\bU}{{\bf U}}
\newcommand{\bV}{{\bf V}}
\newcommand{\bX}{{\bf X}}
\newcommand{\mbD}{{\mathbb{D}}}

\newcommand{\be}{{\bf e}}
\newcommand{\bz}{{\bf z}}
\newcommand{\bx}{{\bf x}}
\newcommand{\bff}{{\bf f}}
\newcommand{\bd}{{\bf d}}
\newcommand{\bo}{{\bf }}
\newcommand{\bnu}{\boldsymbol\nu}
\newcommand{\bn}{\boldsymbol n}
\newcommand{\btau}{\boldsymbol\tau}
\newcommand{\bsigma}{\boldsymbol\sigma}
\newcommand{\bpsi}{\boldsymbol\psi}
\newcommand{\bphi}{\boldsymbol\phi}
\def\Web{\mbox{\text{We}}} 
\def\Fro{\mbox{\text{Fr}}}  
\def\Rey{\mbox{ \text{Re}}}  
\def\Wei{\mbox{ \text{Wi}}} 
 
 
 \newcommand{\mbT}{{\mathbb{T}}}
\newcommand{\mbS}{{\mathbb{S}}}
 
\newcommand{\mbI}{{\mathbb{I}}} 
\newcommand{\mbR}{{\mathbb{R}}}
\newcommand{\mbP}{{\mathbb{P}}}
\newcommand{\mbB}{{\mathbb{B}}}
 \newcommand{\bp }{\boldsymbol{+}}
 \newcommand{\bop }{\boldsymbol{\oplus}}
\newcommand{\bomg}{\boldsymbol\omega}
 
 

\definecolor{darkblue}{rgb}{0,0,1}
\definecolor{darkgreen}{rgb}{0,0.6,0}
\definecolor{darkbrown}{rgb}{0,0.9,0.8}
\definecolor{darkblue&}{rgb}{0,0,1}
\definecolor{red}{rgb}{1,0,0}
\definecolor{lightblue}{rgb}{.9,1,1}



\title{\textcolor{darkgreen}{ParMooN \\ (Parallel Mathematical Object Oriented Numerics)}}

\author[Prof. Sashikumaar Ganesan]
{
\textbf{Sashikumaar Ganesan}
\\ 
    {\tiny\textit{jointly with ...}}
    \\ Manah, Thevin
}

\institute[IISc, India]
{
Computational Mathematics Group\\
Department of Computational and Data Sciences\\
Indian Institute of Science, Bangalore, India\\  
}

\date{Version 1.0 (2019)}
     
\begin{document}

\begin{frame}
% \begin{center}
% \Large
% \bf
%  WCCM XII -- APCOM VI -- SEOUL 2016
% \end{center}
% \vspace{-5mm}
  \titlepage
\end{frame}
\footnotesize

% \begin{frame}
% %   \frametitle{Outline}
%   \tableofcontents[part=1]
%  %\tableofcontents[part=1,pausesections]
% \end{frame}


%\AtBeginSection[]
%{
% \begin{frame}<beamer>
%\frametitle{Outline}
% \tableofcontents[current,currentsection]
% \tableofcontents[part=1]
%\end{frame}
%}


\part<presentation>{Main Talk}
 
\section[Mathematical Model]{Introduction to FEM}
%   \subsection{Motivation}
     \frametitle{Motivation}
 
 \onslide<+->
%\begin{block}{Fluid Flow Models}
%\begin{itemize}
%  \item Advanced approaches exists for fixed domains
%\onslide<+->
%  \item Situation changes if the {\color<2>[rgb]{1,0,0}{domain moves} }
%        with {\color<2>[rgb]{1,0,0}{large deformation} }
%  \onslide<+->
%  \item More complicate if the {\color<3>[rgb]{1,0,0}{boundary is priori unknown}}
%        and/or the   {\color<3>[rgb]{1,0,0}{surface tension}} and other effects
%        are considered  
%\end{itemize}
%\end{block}
  \onslide<+->
\begin{block}{Objective}
\begin{itemize}
  \item To develop a robust and reliable algorithms for 
         {\color<4>[rgb]{1,0,0}incompressible flows in moving domains}
\end{itemize}
\end{block}
  \subsection[Model Problem]{Fluid Flow That We Considered}
 % \input{sub03}

  \subsection[Mathematical Model]{Mathematical Model}


\section[Numerical Schemes]{Numerical Schemes}
  \subsection{Weak Formulation}
  
  
\end{document}
